% Kun käytät tätä, älä lataa pakettia stfloats tai fix2col. Tämä lataa
% myös paketin fixltx2e.
%% "When the new output routine for LaTeX3 is done, this package will
%% be obsolete. The sooner the better..."
\RequirePackage{dblfloatfix}

% Kun käytät tätä, et enää tarvitse paketteja type1cm ja type1ec:
\RequirePackage{fix-cm}

% Luokkamäärittelyt.
\documentclass[a4paper,onecolumn,11pt,finnish,twoside,final,oldtoc]{boek3}
\usepackage{calc} % geometry-paketti hyötyy tästä
\usepackage[papersize={185mm,240mm},left=36mm,right=12mm,top=10mm,bottom=25mm,%
            asymmetric,bindingoffset=10mm,marginparwidth=30mm]{geometry}
\reversemarginpar % marginaalit toisin päin
\setcounter{secnumdepth}{7} % kaikki numeroidaan ilman *-optiota

% alleviivaus yms
\usepackage[normalem]{ulem}

% "Extended conditional commands"
\usepackage{xifthen}

% Erikoismerkit yms.
\usepackage{texnames}

% AMS & matematiikka
\usepackage{mathtools}
\usepackage{amssymb}
\usepackage{amsthm}

% LuaLaTeX
\usepackage{cmap}
\usepackage{fontspec}
\usepackage{unicode-math}
\defaultfontfeatures{Ligatures=TeX}
\setmainfont[Extension=.otf,UprightFont=*_R,ItalicFont=*_RI,BoldFont=*_RB,BoldItalicFont=*_RBI]{LinLibertine}
%\setmainfont[Scale=0.92]{Heuristica} % classic
%\setmainfont{Kameron} % contemporary
\setmathfont{xits-math.otf}
\usepackage[protrusion=true,expansion=true,verbose=true]{microtype}

% Otsikoiden säätelyyn
\usepackage{titlesec}

\makeatletter
\g@addto@macro\verbatim{\pdfprotrudechars=0 \pdfadjustspacing=0\relax}
\makeatother

% Babel
\usepackage[finnish,english]{babel}

% Tee jotain jokaisen sivun kohdalla (totpages tarvitsee tätä):
\usepackage{everyshi}

% Riviväli 1,5
\usepackage{setspace}
\onehalfspacing

% URL:ien ladonta ja "tavutus"
\usepackage[obeyspaces,spaces,hyphens,T1]{url}

% TikZ-paketti
\usepackage{pgf,tikz}
\usetikzlibrary{arrows}

% Tunnisteiden ja sivunumeroiden asettamista varten
\usepackage{fancyhdr}

% VerbatimOut Pythonia varten
\usepackage{fancyvrb}
\usepackage{fancybox}

% Euron merkki
\usepackage{eurosym}

%% Hymiö
%\usepackage{wasysym}

% Harjoitustehtävät answers-paketilla
\usepackage{answers}

% Cancel-paketti supistamista ym. varten
\usepackage{cancel}

% Monipalstainen sisältö esim. lyhyille tehtäville
% Käyttö: \begin{multicols}{n}
%              ...
%         \end{multicols}
\usepackage{multicol}

% Tuki numeroiduille listoille
\usepackage{enumerate}

% Desimaalipilkut
% Käyttö: väli pilkun jälkeen -> listapilkku; ei väliä pilkun jälkeen -> desimaalipilkku
\usepackage{icomma}

%% Ei-kursivoidut kreikkalaiset aakkoset, esim. $\upmu$
%\usepackage{upgreek}

% Automatisoitu hakemisto
\usepackage{makeidx}
\makeindex

% Aika- ja pvmformaattien kustomointiin
\usepackage{datetime}

% Täältä saa komennon \ifthispageodd
\usepackage{scrextend}

% Marginaalien muuttaminen harjoitustehtäväsivuja varten
\usepackage[strict]{changepage}

% Käännetyt tekstit taulukoita varten
\usepackage{rotating}

% Taulukot
\usepackage{array}

%% Testailuun, komentoja: \blindtext, \Blindtext, \pagevalues
%\usepackage{blindtext}
%\usepackage{layouts}

% Ei sisennetä kappaleen ekaa riviä
\setlength{\parindent}{0.0cm}

%\theoremstyle{definition}
%\newtheorem{theorem}{Teoreema}

% Lisäattribuutteja \includegraphics-komentoon
\usepackage{graphicx}

% allaolevia loogisia muuttujia voi muuttaa meta.tex-tiedostossa
% esim. \varitfalse

% väreissä vai mustavalkoisena
% vaatii vastaavat kuvatiedostot
% default: väreissä
\newif\ifvarit
\varittrue

\newcommand{\kurssinTunnus}{}
\newcommand{\kurssinNimi}{}
\newcommand{\sitaatti}{Sitaatti puuttuu.}
\newcommand{\sitaatinLahde}{}
\newcommand{\kirjanVersio}{}
\newcommand{\kirjanPainopaikka}{}
\newcommand{\kirjanVastaavuus}{}
\newcommand{\metasivu}{}

% Hypersetup (metatason pdf-säätöä)
\input{config/hypersetup}

% Omat komennot
% pitkän matematiikan kursseja
\newcommand{\maaI}{Funktiot ja yhtälöt}
\newcommand{\maaII}{Polynomifunktiot}
\newcommand{\maaIII}{Geometria}
\newcommand{\maaIV}{Analyyttinen geometria}
\newcommand{\maaV}{Vektorit}
\newcommand{\maaVI}{Todennäköisyys ja tilastot}
\newcommand{\maaVII}{Derivaatta}
\newcommand{\maaVIII}{Juuri- ja logaritmifunktiot}
\newcommand{\maaIX}{Trigonometriset funktiot ja lukujonot}
\newcommand{\maaX}{Integraalilaskenta}
\newcommand{\maaXI}{Lukuteoria ja logiikka}
\newcommand{\maaXII}{Numeerisia ja algebrallisia menetelmiä}
\newcommand{\maaXIII}{Differentiaali- ja integraalilaskennan jatkokurssi}
\newcommand{\maaXIV}{Kertaus}

% lyhyen matematiikan kursseja
\newcommand{\mabI}{Lausekkeet ja yhtälöt}
\newcommand{\mabII}{Geometria}
\newcommand{\mabIII}{Matemaattisia malleja I}
\newcommand{\mabIV}{Matemaattinen analyysi}
\newcommand{\mabV}{Tilastot ja todennäköisyys}
\newcommand{\mabVI}{Matemaattisia malleja II}
\newcommand{\mabVII}{Talousmatematiikka}
\newcommand{\mabVIII}{Matemaattisia malleja III}
\newcommand{\mabIX}{Kertaus}

% nuolia
\newcommand{\ekvi}{\Longleftrightarrow}
\newcommand{\impli}{\Longrightarrow}

% sanoja math-moodissa
\newcommand{\kun}{\textnormal{kun} \;}
\newcommand{\jos}{\textnormal{jos} \;}
\newcommand{\tai}{\textnormal{tai} \;}

% yleisimpiä joukkoja
\newcommand{\N}{\mathbb{N}}
\newcommand{\Z}{\mathbb{Z}}
\newcommand{\Q}{\mathbb{Q}}
\newcommand{\R}{\mathbb{R}}
\newcommand{\C}{\mathbb{C}}

\input{config/meta}

\ifvarit
    \newcommand{\vari}{}
    \definecolor{vapaa_matikka_vari_3}{cmyk}{0.2,0.27,0.0,0.0}	% Laatikot
    \definecolor{vapaa_matikka_vari_4}{cmyk}{1.0,0.42,0.65,0.0}	% Esimerkit
    \definecolor{vapaa_matikka_vari_5}{cmyk}{0.60,0.80,0.0,0.0}	% Tehtävänumerot
\else
    \newcommand{\vari}{_grey}
    \definecolor{vapaa_matikka_vari_3}{cmyk}{0.0,0.0,0.0,0.25}	% Laatikot
    \definecolor{vapaa_matikka_vari_4}{cmyk}{0.0,0.0,0.0,0.6}	% Esimerkit
    \definecolor{vapaa_matikka_vari_5}{cmyk}{0.0,0.0,0.0,0.8}	% Tehtävänumerot
\fi

% Tarvitaan kuvien ja taulukkojen vierekkäin laittamiseen.
\def\vcent#1{\mathsurround0pt$\vcenter{\hbox{#1}}$}

% Lisäfontteja
\newfontface\fontI[Scale=3.1]{OpenSans-Light.ttf}
\newfontface\fontII[Scale=2.3]{OpenSans-Light.ttf}
\newfontface\fontIII[Scale=1.5]{OpenSans-Light.ttf}
\newfontface\fontIV[Scale=0.9]{OpenSans-Regular.ttf}
\newfontface\fontV[Scale=1.2]{OpenSans-Regular.ttf}
% contemporary
% (Thin - ExtraLight - Light - Regular - Medium - SemiBold - Bold - ExtraBold - Black)
%\newfontface\fontI[Scale=3.1]{Raleway-Light.otf}
%\newfontface\fontII[Scale=2.3]{Raleway-ExtraLight.otf}
%\newfontface\fontIII[Scale=1.5]{Raleway-Light.otf}
%\newfontface\fontIV[Scale=0.9]{Raleway-Medium.otf}
%\newfontface\fontV[Scale=1.2]{Raleway-Regular.otf}

\usepackage{../commons/packages/esimerkki}
\usepackage{../commons/packages/kansilehti}
\usepackage{../commons/packages/kuva}
\usepackage{../commons/packages/kuvaaja}
\usepackage{../commons/packages/laatikot}
\usepackage{../commons/packages/logiikka}
\usepackage{../commons/packages/lukusuora}
\usepackage{../commons/packages/merkkikaavio}
\usepackage{../commons/packages/otsikkotyylit}
\usepackage{../commons/packages/ppalkki}
\usepackage{../commons/packages/qrlinkki}
\usepackage{../commons/packages/tehtavat}
\usepackage{../commons/packages/termi}
\usepackage{../commons/packages/todistus}
\usepackage{../commons/packages/tunnisteet}
\usepackage{../commons/packages/valit}


\let\cleardoublepage\clearpage
